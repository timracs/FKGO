\documentclass{article}
\usepackage[T1]{fontenc}                
\usepackage[utf8]{inputenc} 
\usepackage[ngerman]{babel} 
\usepackage{enumerate}
\usepackage{geometry}
\usepackage{titlesec}
\usepackage{hyperref}
\usepackage{ifthen}
\usepackage{color}
\usepackage[official]{eurosym}

\setlength\parindent{0pt}
\DeclareUnicodeCharacter{20AC}{\euro}
\geometry {a4paper, top= 25mm, bottom=25mm, left=25mm, right=25mm}
\titleformat{\part}{\fontsize{15pt}{15pt}\bfseries}{\thepart .\ }{0pt}{}
\titleformat{\chapter}{\fontsize{15pt}{15pt}\bfseries}{\thechapter .\ }{0pt}{}
\titleformat{\section}{\fontsize{11pt}{13pt}\bfseries}{\S \ \thesection \ }{0pt}{\normalsize}

\begin{document}
\begin{center}
    \huge \textbf{Anlage I zur FKGO -- AFSG}
\end{center}


\section{Summe der AFsG}
\begin{enumerate}[(1)]
    \item Der im Haushaltsplan vorgesehene Posten für AFsG eines endenden Semsters beträgt, soweit nichts anderes bestimmt ist, 40.000€.
    \item Im Haushaltsplan 2019/20 beträgt der gemeinsame Posten für die AFsG der Semester Wintersemester 2019/20 und Sommersemester 2019 35.000€.
    \item Für die Berechnung der AFSG der Semester Wintersemester 2019/20 und Sommersemester 2019 wird abwischend von Abs.\ 2 mit einem Betrag von jeweils 40.000€ gerechnet.
    \item Zur Berechnung der Beträge der AFsG der Semester Wintersemester 2019/20 und Sommersemester 2019 im Haushaltsplan 2020/21 wird abwischend von Abs.\ 2 von einem Betrag von jeweils 40.000€ ausgegangen.
\end{enumerate}

\section{AFsG-Sockelsatz}
\begin{enumerate}[(1)]
	\item Soweit nicht anders bestimmt, beträgt der AFsG-Sockelsatz 1000 €.
    \item Für das Sommersemester 2019 und das Wintersemester 2019/20 beträgt der AFsG-Sockelsatz 500 €.
\end{enumerate}
\end{document}
