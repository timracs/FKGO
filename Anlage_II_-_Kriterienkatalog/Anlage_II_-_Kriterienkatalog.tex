\documentclass{article}
\usepackage[T1]{fontenc}                
\usepackage[utf8]{inputenc} 
\usepackage[ngerman]{babel} 
\usepackage{enumerate}
\usepackage{geometry}
\usepackage{titlesec}
\usepackage{hyperref}
\usepackage{ifthen}
\usepackage{color}
\usepackage{tabularx}
\usepackage[official]{eurosym}

\setlength\parindent{0pt}
\DeclareUnicodeCharacter{20AC}{\euro}
\geometry {a4paper, top= 25mm, bottom=25mm, left=25mm, right=25mm}
\titleformat{\part}{\fontsize{15pt}{15pt}\bfseries}{\thepart .\ }{0pt}{}
\titleformat{\chapter}{\fontsize{15pt}{15pt}\bfseries}{\thechapter .\ }{0pt}{}
\titleformat{\section}{\fontsize{12pt}{13pt}\bfseries}{\thesection .\ }{0pt}{}

\newcolumntype{L}[1]{>{\raggedright\arraybackslash}p{#1}} % linksbündig mit Breitenangabe
\newcolumntype{C}[1]{>{\centering\arraybackslash}p{#1}} % zentriert mit Breitenangabe
\newcolumntype{R}[1]{>{\raggedleft\arraybackslash}p{#1}} % rechtsbündig mit Breitenangabe

\begin{document}
\noindent
\begin{center}
    \huge \textbf{Anlage II zur FKGO -- Kriterienkatalog}
\end{center}

\section{Allgemeine Regelungen}
    \begin{itemize}
        \item Für Jährliche Maximalbeträge beginnt der Berechnungszeitraum mit dem Wintersemester (1.\ Oktober) und endet mit dem Sommersemester (30.\ September).
        \item Die Semester erstrecken sowohl über die Vorlesungszeit, als auch über die darauf folgende vorlesungsfreie Zeit.
        \item Anträgen sind mit den vom Fachschaftenkollektiv bereitgestellten, vollständig ausgefüllten Antragsformularen einzureichen. 
        	Die Einreichung soll digital per E-Mail erfolgen.
        \item Allen Anträgen müssen die folgendem Dokumente beigefügt werden:
        \begin{itemize}
            \item tabellarische Kostenkalkulation
            \item \textbf{Kopien} aller Rechnungen
            \item Arbeitsberichte (außer bei  "`Sonstiges"'). 
            	Der Arbeitsbericht muss dokumentieren, welche Awareness-Maßnahmen getroffen wurden.
        \end{itemize}
    \end{itemize}

\section{Veranstaltungen}
    Insgesamt können maximal \textbf{2500 € im Jahr} für Veranstaltungen beantragt werden. \\
    
    \setlength\extrarowheight{2mm} \sffamily    
    \begin{tabular}{l||L{6cm}|L{7cm}}

         &\textbf{Titel}  & \textbf{Höchstsatz} \\[1mm] \hline \hline
         a&
         Erstsemesterarbeit &
         800 € pro Semester\par
         davon für Verpflegung max.\ 400 € \\[1mm] \hline
          
        b&
        Inhaltliche Veranstaltungen & 
        700 € pro Veranstaltung \\[1mm]
        
    \end{tabular}
    \rmfamily
    \subsection{Anmerkungen}
    Anträgen muss zusätzlich beigefügt werden:
    \begin{itemize}
        \item Veranstaltungsprogramm
    \end{itemize}
        \subsubsection{Erstsemesterarbeit}
        \begin{itemize}
            \item Veranstaltungen müssen im zeitlichen Zusammenhang zum Semesterbeginn/Einschreibung (ca.\ 1 Monat) stehen.
            \item Im Sommersemester ist das Vorhandensein von Erstsemestern nachzuweisen.
            \item Verpflegung ist nur bis zum angegebenen Höchsbetrag förderbar. 
            	Alkohol wird nicht erstattet.
        \end{itemize}
        
        \subsubsection{Inhaltliche Veranstaltungen}   
        Der Posten inhaltliche Veranstaltungen dient der Förderung von Veranstaltungen, die der fachlichen und allgemeinen Bildung der Mitglieder der Fachschaft dienen.
        Zur Förderung sollte die Veranstaltung die folgenden Voraussetzungen erfüllen:
        \begin{itemize}
            \item Sie bietet Vorteile für das Studium in Form von Fachinformationen oder -methoden.
            \item Die Veranstaltung hat eine formelle Struktur und ein Programm.
            \item Sie dient der strukturierten Weitergabe von akademischen Informationen, z.B.\ in Form eines Vortrags oder es besteht ein organisiertes Debattenformat, z.B.\ in Form einer Podiumsdiskussion.
        \end{itemize}

        Nicht als inhaltlich förderbar sind solche Veranstaltungen, die in ihrer Natur vordergründig dem sozialen Austausch der Teilnehmer untereinander dienen, zum Beispiel:
        \begin{itemize}
            \item Brettspieleabende,
            Paintball,
            Veranstaltungen mit dem Ziel sich zu betrinken,
            Partys,
            Bälle,
            (saisonale) Feiern
            \item Verpflegung und Alkohol sind nicht förderbar
        \end{itemize}
        





\section{Allgemeine Fahrten}
    Insgesamt können maximal \textbf{4000 € im Jahr} für Allgemeine Fahrten beantragt werden. \newline
    Es können je Person und Tag \textbf{maximal 50 €} beantragt werden.\\
    
    \setlength\extrarowheight{2mm} \sffamily    
    \begin{tabular}{l||L{6cm}|L{7cm}}
    
    	&\textbf{Titel} &  \textbf{Höchstsatz}\\ \hline \hline
    	c &
    	Erstsemesterfahrten &
    	
    	bis zu 30 Teilnehmer: 800 € \newline
    	bis zu 50 Teilnehmer: 900 € \newline
    	über 50 Teilnehmer: 1000 €\\ \hline
    
    	d &
    	Klausurfahrten &
    	bis zu 10 Teilnehmer: 500 € \newline
    	bis zu 30 Teilnehmer: 800 € \newline
    	über 30 Teilnehmer: 900 €  \\ \hline
    	
    	e &
    	BuFaTa &
    	800 € \\
    \end{tabular}
    \rmfamily
    \subsection{Anmerkungen}
        \begin{itemize}
            \item Dem Antrag muss eine Teilnehmerliste mit Unterschriften aller Teilnehmer beigefügt werden.
            \item Anträge für Fahrten ins Ausland bedürfen einer Vorankündigung.
            \item Verpflegung und Alkohol sind nicht förderbar.
        \end{itemize}
        \subsubsection{Erstsemesterfahrt}
            \begin{itemize}
                \item  Eine Erstsemesterfahrt soll sich gezielt an Erstsemester richten.
                Der Anteil an Nicht-Studienanfängern soll 30 \% nicht überschreiten.
                \item Erstsemester im Sinne dieser Ordnung sind alle Studierende, die in mindestens einem Studiengang noch nicht länger als ein Studienjahr an der RFWU Bonn eingeschrieben sind.
            	\item Verpflegung und Alkohol sind nicht förderbar.
            \end{itemize}
        \subsubsection{Klausurfahrten}
            \begin{itemize}
                \item  Fahrt für aktive Fachschaftsmitglieder, um gezielt an fachschafts- bezogenen Themen zu arbeiten 
                \item Klausurfahrten sollen regional im Umkreis zu Bonn stattfinden. (< 100 km Entfernung)
                \item Verpflegung und Alkohol sind nicht förderbar.
            \end{itemize}
        \subsubsection{Teilnahme BuFaTa}
            \begin{itemize}
                \item Teilnahme an landes-, bundes-, europa- oder weltweiten Fachschaftsversammlungen
            \end{itemize}

\newpage

 \section{Exkursionen}
    Insgesamt können maximal \textbf{2000 € im Jahr} für Exkursionen beantragt werden. \\
    
    \setlength\extrarowheight{2mm} \sffamily
    \begin{tabular}{l||L{6cm}|L{7cm}}
        &\textbf{Titel} &  \textbf{Höchstsatz}\\[1mm] \hline \hline
        f&
        Bildungsfahrt &
        bis zu 20 Teilnehmer: 800 € \newline
        bis zu 50 Teilnehmer: 900 € \newline
        über 50 Teilnehmer: 1000 € \\[1mm] \hline
         
        g&
        Tagesexkursion &
        bis zu 20 Teilnehmer: 100 € \newline
        bis zu 50 Teilnehmer: 250 € \newline
        über 50 Teilnehmer: 500 € \\[1mm]
    \end{tabular}
    \rmfamily
    \subsection{Anmerkungen}
    Dem Antrag muss zusätzlich beigefügt werden:
        \begin{itemize}
            \item Programm der Fahrt/Exkursion
            \item Teilnehmerliste mit Unterschriften aller Teilnehmer
        \end{itemize}
        \subsubsection{Bildungsfahrt}
            \begin{itemize}
                \item  mehrtägige Exkursion mit Fachbezug, offen für die gesamte Fachschaft
                \item Verpflegung und Alkohol sind nicht förderbar.
            \end{itemize}
        
        \subsubsection{Tagesexkursion}
            \begin{itemize}
                \item eintägige Exkursion mit Fachbezug, offen für die gesamte Fachschaft
                \item Verpflegung und Alkohol sind nicht förderbar.
            \end{itemize}
            

    
\section{Sonstiges}
    \setlength\extrarowheight{2mm} \sffamily
    \begin{tabular}{l||L{6cm}|L{7cm}}
        &\textbf{Titel} & \textbf{Höchstsatz}\\[1mm] \hline \hline
        h&
        Computer und Zubehör &
        400 € pro Jahr \\[1mm] \hline
         

         
        i&
        Ausrichtung BuFaTa  &
        2000 € pro BuFaTa \\[1mm] \hline
        
        j&
        Fachschaftskleidung&
        400 € pro Jahr \\[1mm] \hline
         
        k&
        Fachschaftsneugründung &
        AFsG-Sockelsatzes, i.d.R.\ 1000 € \\[1mm]
    \end{tabular}
    \rmfamily
    \subsection{Anmerkungen}
        \subsubsection{Computer und Zubehör}
            \begin{itemize}
                \item  Reparatur oder Kauf von EDV-Geräten, kein alltäglicher Bürobedarf wie Druckerpatronen
            \end{itemize}
        \subsubsection{Ausrichtung BuFaTa}
            \begin{itemize}
                \item  Ausrichtung sowie Vor- und Nachbereitung von landes-, bundes-, europa- oder weltweiten Fachschaftsversammlungen in Bonn.
            \end{itemize}
        \subsubsection{Fachschaftsneugründung}
            \begin{itemize}
                \item Maßgeblich ist die Aufnahme in die Fachschaftenliste (Anhang \glqq Fachschaftenliste\grqq\ FKGO).
            \end{itemize}

\pagebreak
\end{document}
